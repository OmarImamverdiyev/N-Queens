\documentclass[11pt]{article}
\usepackage[a4paper,margin=1in]{geometry}
\usepackage{amsmath,amssymb,amsthm}
\usepackage{algorithm}
\usepackage{algpseudocode}
\usepackage{booktabs}
\usepackage{hyperref}
\usepackage{enumitem}

\title{N-Queens as a Constraint Satisfaction Problem:\\Mathematical Formulation and Solution Methods}
\author{}
\date{\today}

\begin{document}
\maketitle

\begin{abstract}
This document presents a mathematical, implementation-independent view of the N-Queens problem as a Constraint Satisfaction Problem (CSP). It explains the CSP model, iterative local search (min-conflicts), the MRV and LCV heuristics, tie-breaking rules, and AC-3 constraint propagation. Worked examples are included to show how each method operates in practice.
\end{abstract}

\section{Problem Statement}
Given an integer $n$, place $n$ queens on an $n \times n$ chessboard such that no two queens attack each other.

Project constraints:
\begin{itemize}[leftmargin=*]
  \item $10 \le n \le 1000$.
  \item Input is a file with $n$ lines; line $i$ stores the column of the queen in row $i$.
  \item Comments and blank lines may be present in the file.
\end{itemize}

The required algorithmic ingredients are:
\begin{enumerate}[leftmargin=*]
  \item A CSP search algorithm.
  \item Heuristics: MRV, LCV, tie-breaking.
  \item Constraint propagation via AC-3.
\end{enumerate}

\section{CSP Formulation}
\subsection{Variables and Domains}
Use one variable per row:
\[
X_i \in \{0,1,\dots,n-1\}, \quad i=0,\dots,n-1,
\]
where $X_i$ is the column chosen for row $i$.

\subsection{Constraints}
For all pairs $i \ne j$:
\begin{align}
X_i &\ne X_j \quad \text{(different columns)}, \\
|X_i - X_j| &\ne |i - j| \quad \text{(different diagonals)}.
\end{align}

So each pair of rows is linked by a binary constraint. The CSP graph is complete.

\subsection{Objective View for Local Search}
For local search, define conflict count:
\[
F(X) = \sum_{0 \le i < j < n}
\mathbf{1}[X_i = X_j \ \lor\ |X_i-X_j| = |i-j|].
\]
Any assignment with $F(X)=0$ is a valid N-Queens solution.

\section{Iterative Local Search: Min-Conflicts}
\subsection{Why Iterative Search Fits Large $n$}
Backtracking is complete but can be expensive on large boards. Min-conflicts scales better in practice for large $n$, especially with good heuristics and restarts.

\subsection{General Procedure}
\begin{algorithm}[h]
\caption{Iterative Min-Conflicts for N-Queens}
\begin{algorithmic}[1]
\State Initialize a complete assignment $X$ (typically random, one queen per column)
\For{$t = 1$ to \textit{maxSteps}}
  \If{$F(X)=0$}
    \State \Return $X$
  \EndIf
  \State Select a conflicted row $r$ using a row-selection heuristic (MRV + tie-break)
  \State Select a column $c$ for row $r$ using LCV (with tie-breaking)
  \State Set $X_r \gets c$
  \State Optionally use random restart if progress stalls
\EndFor
\State \Return failure (step limit reached)
\end{algorithmic}
\end{algorithm}

\section{Heuristics}
\subsection{MRV: Minimum Remaining Values}
MRV chooses the row with smallest current feasible domain:
\[
r^* = \arg\min_{r \in U} |D_r|,
\]
where $U$ is the set of candidate rows and $D_r$ is row $r$'s current domain after propagation.

Intuition: prioritize the most constrained variable first.

\subsection{LCV: Least Constraining Value}
For a selected row $r$, LCV chooses the column that eliminates the fewest values from neighboring domains:
\[
\text{LCVScore}(r,c) =
\sum_{j \in N(r)}
\left| \left\{ d \in D_j \mid (r,c) \text{ conflicts with } (j,d) \right\} \right|.
\]
Choose $c$ minimizing $\text{LCVScore}(r,c)$.

Intuition: leave maximal flexibility for future decisions.

\subsection{Tie-Breaking}
If MRV or LCV has ties, apply explicit tie-break rules. Common choices:
\begin{itemize}[leftmargin=*]
  \item Deterministic: smaller row index, smaller column index.
  \item Conflict-aware: prefer row with larger current conflict count.
  \item Stochastic: random among exact ties.
\end{itemize}
Tie-breaking stabilizes behavior and helps avoid cyclic behavior.

\section{Constraint Propagation with AC-3}
\subsection{Arc Consistency}
An arc $(X_i, X_j)$ is consistent if for every value $a \in D_i$, there exists some $b \in D_j$ such that constraints between $X_i$ and $X_j$ are satisfied.

\subsection{Revise Step}
\[
\text{Revise}(D_i,D_j): \quad
D_i \gets D_i \setminus
\{a \in D_i \mid \nexists b \in D_j \ \text{compatible}(a,b)\}.
\]
If $D_i$ changes, neighboring arcs must be reconsidered.

\subsection{AC-3 Algorithm Sketch}
\begin{algorithm}[h]
\caption{AC-3}
\begin{algorithmic}[1]
\State Initialize queue with arcs $(X_i,X_j)$, $i \ne j$
\While{queue not empty}
  \State Pop $(X_i,X_j)$
  \If{\textsc{Revise}$(D_i,D_j)$}
    \If{$D_i = \varnothing$}
      \State \Return inconsistent
    \EndIf
    \For{each neighbor $X_k$ of $X_i$, $k \ne j$}
      \State Add $(X_k,X_i)$ to queue
    \EndFor
  \EndIf
\EndWhile
\State \Return arc-consistent domains
\end{algorithmic}
\end{algorithm}

\section{Worked Examples}
\subsection{Example 1: CSP Constraints for $n=4$}
Variables: $X_0,X_1,X_2,X_3 \in \{0,1,2,3\}$.

One valid solution is:
\[
(X_0,X_1,X_2,X_3)=(1,3,0,2).
\]
Check quickly:
\begin{itemize}[leftmargin=*]
  \item All columns are distinct.
  \item For every pair $(i,j)$, $|X_i-X_j| \ne |i-j|$.
\end{itemize}
So this assignment satisfies all constraints.

\subsection{Example 2: One AC-3 Revise Operation}
Suppose in some step:
\[
D_1 = \{2\}, \quad D_3 = \{0,2,4\}.
\]
Rows differ by $|3-1|=2$.

If $X_1=2$, then for row $3$:
\begin{itemize}[leftmargin=*]
  \item Column $2$ is forbidden (same column).
  \item Columns $0$ and $4$ are forbidden (diagonals: $2\pm2$).
\end{itemize}
So every value in $D_3$ loses support, and
\[
D_3 \gets \varnothing.
\]
AC-3 immediately detects inconsistency in this branch/state.

\subsection{Example 3: MRV + LCV + Tie-Break}
Assume active rows $\{2,5,7\}$ with domains:
\[
|D_2|=2,\quad |D_5|=3,\quad |D_7|=2.
\]
MRV tie is between rows $2$ and $7$.

If tie-break says ``pick row with larger current conflicts'' and row $7$ is worse, choose row $7$.

Now for row $7$, suppose:
\[
\text{LCVScore}(7,2)=3,\quad \text{LCVScore}(7,5)=1.
\]
Choose column $5$ because it constrains neighbors less.

\subsection{Example 4: Min-Conflicts Move}
For $n=8$, start with diagonal assignment:
\[
X=[0,1,2,3,4,5,6,7].
\]
Each queen has severe diagonal conflicts. If we select row $3$ and evaluate candidate columns, moving to a low-conflict column (for example one with only $1$ conflict instead of $7$) decreases $F(X)$ sharply. Repeating this process usually converges quickly with restarts when plateaus occur.

\section{Alternative Method: Full Backtracking CSP}
Another valid CSP approach is complete search:
\begin{itemize}[leftmargin=*]
  \item Backtracking assignment by assignment.
  \item MRV for variable order.
  \item LCV for value order.
  \item AC-3 after each assignment.
\end{itemize}
This method is complete (finds a solution if one exists), but for large $n$ it is generally less practical than iterative min-conflicts.

\section{Complexity Notes}
\begin{itemize}[leftmargin=*]
  \item N-Queens has $n$ variables and domain size up to $n$.
  \item The binary-constraint graph is dense ($\Theta(n^2)$ arcs).
  \item AC-3 improves pruning but has overhead, so in iterative solvers it is often applied selectively.
  \item Min-conflicts is incomplete but empirically strong for large N-Queens.
\end{itemize}

\section{Mapping to Project Requirements}
\begin{center}
\begin{tabular}{@{}ll@{}}
\toprule
Requirement & Mathematical method \\
\midrule
Iterative search & Min-conflicts local search \\
MRV heuristic & Choose row with smallest propagated domain \\
LCV heuristic & Choose value minimizing neighbor eliminations \\
Tie-breaking rules & Deterministic and/or randomized exact-tie rules \\
Constraint propagation & AC-3 on row-domain arcs \\
\bottomrule
\end{tabular}
\end{center}

\section{Conclusion}
N-Queens fits naturally as a CSP with binary non-attacking constraints. For large-board constraints such as $10 \le n \le 1000$, an iterative min-conflicts method augmented with MRV, LCV, tie-breaking, and AC-3 offers a practical balance between CSP rigor and scalability.

\end{document}
